\section{From \sw to \asw}
\label{sec:asofte}

Software engineering (SE) research investigates questions pertaining to the design, development, maintenance, testing, and evolution of software systems. As software continues to pervade a wide range of industries, both open- and closed-source code repositories have grown to become unprecedentedly large and complex. This has resulted in an increase of unstructured, unlabeled, yet important data including requirements, design documents, source code files, test cases, and defect reports. Previously, the software engineering community has applied canonical Artificial Intelligence (AI) techniques to identify patterns and unique relationships within this data to automate or enhance many tasks typically performed manually by developers. Unfortunately, the process of implementing canonical AI techniques can be a tedious exercise in careful feature engineering, wherein researchers experiment with identifying salient attributes of data that can be leveraged to help solve a given problem or automate a given task.

However, with recent improvements in computational power and the amount of memory available in modern computer architectures, an advancement to traditional AI approaches has arisen called Deep Learning (DL). Deep learning represents a fundamental shift in the manner by which machines learn patterns from data by \textit{automatically} extracting salient features for a given computational task as opposed to relying upon human intuition. Deep Learning approaches are characterized by architectures comprised of several layers that perform mathematical transformations on data passing through them. These transformations are controlled by sets of learnable parameters that are adjusted using a variety of learning and optimization algorithms. These computational layers and parameters form models that can be trained for specific tasks by updating the parameters according to a model's performance on a set of training data. Given the immense amount of structured and unstructured data in software repositories that are likely to contain hidden patterns, DL techniques have ushered in advancements across a range of tasks in software engineering research including automatic program repair~\citep{Tufano2018}, code suggestion~\citep{Gu2018}, defect prediction~\citep{Wang2016}, malware detection \cite{Li2018}, feature location~\citep{Corley2015}, among many others~\citep{Ma2018, Wan2018, Liu2018, White2016, Xu2016, Guo2017, Tian2018a, Liu2017}. A recent report from the 2019 NSF Workshop on Deep Leaning \& Software Engineering has referred to this area of work as Deep Learning for Software Engineering (\dlse)~\citep{dlse19-report}. 

The applications of DL to improve and automate SE tasks points to a clear synergy between ongoing research in SE and DL. However, in order to effectively chart the most impactful path forward for research at the intersection of these two fields, researchers need a clear map of what has been done, what has been successful, and what can be improved. %
In an effort to map and guide research at the intersection of DL and SE, we conducted a systematic literature review (SLR) to identify and systematically enumerate the synergies between the two research fields. As a result of the analysis performed in our SLR, we synthesize a detailed \textit{research roadmap} of past work on DL techniques applied to SE tasks\footnote{It should be noted that another area, known as Software Engineering for Deep Learning (SE4DL), which explores improvements to engineering processes for DL-based systems, was also identified at the 2019 NSF workshop. However, the number of papers we identified on this topic was small, and mostly centered around emerging testing techniques for DL models. Therefore, we reserve a survey on this line of research for future work.} (\ie \dlse), complete with identified open challenges and best practices for applying DL techniques to SE-related tasks and data. Additionally, we analyzed the impacts of these DL-based approaches and discuss some observed concerns related to the potential reproducibility and replicability of our studied body of literature. % 