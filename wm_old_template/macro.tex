\makeatletter

% this  is a easy way to add and highlight new text  ...
% just comment in/out the \tnew macro ..

\newcommand{\tnew}[1]{{\bf { #1 }} }
%\newcommand{\tnew}[1]{{ { #1 }} }

% math and theorem definition

\newcommand{\ndef}{\stackrel{\rm def}{=}}

% this is used for draft only

%\renewcommand{\baselinestretch}{2}

% just to number pages in the draft

\useunder{\uline}{\ul}{}

% nothing i.e., no-numbering final and camera ready

%\pagestyle{empty}

\newcommand{\CrashScopesp}{{\textsc CrashScope~}}
\newcommand{\CrashScopes}{{\textsc \small CrashScope's~}}
\newcommand{\CrashDroid}{{\textsc CrashDroid~}}
\newcommand{\CrashScopebf}{{ \textbf{\textsc CrashScope~}}}

\newcommand{\ReDraw}{{\textsc{\small ReDraw}}\xspace}
\newcommand{\ReDraws}{{\textsc{\small ReDraw's}}\xspace}
\newcommand{\Remaui}{{\textsc{\small Remaui}}\xspace}
\newcommand{\Remauis}{{\textsc{\small Remaui's}}\xspace}
\newcommand{\CrashScope}{{\textsc{\small CrashScope}}\xspace}
\newcommand{\MonkeyLab}{{\textsc{\small MonkeyLab}}\xspace}
\newcommand{\pixcode}{{pix2code}\xspace}

\newcommand{\GVTsp}{{\textsc{\small Gvt~}}}
\newcommand{\GVT}{{\textsc{\small Gvt}}}
\newcommand{\GVTs}{{\textsc{\small Gvt's~}}}
\newcommand{\dv}{{\textit{DV}}\xspace}
\newcommand{\dvs}{{\textit{DVs}}\xspace}
\newcommand{\gc}{{\textit{GC}}\xspace}
\newcommand{\gcs}{{\textit{GCs}}\xspace}
\newcommand{\mgc}{{\textit{M-GC}}\xspace}
\newcommand{\mgcs}{{\textit{M-GCs}}\xspace}
\newcommand{\igc}{{\textit{I-GC}}\xspace}
\newcommand{\igcs}{{\textit{I-GCs}}\xspace}


\newcommand{\ReDrawAbs}{{R}{\ssmall E}{D}{\ssmall RAW}\xspace}




\newsavebox\CBox
\newlength\CLength
%\def\circled#1{\sbox\CBox{#1}%
%  \ifdim\wd\CBox>\ht\CBox \CLength=\wd\CBox\else\CLength=\ht\CBox\fi
%    \makebox[1.2\CLength]{\makebox(0,0.9\CLength){\put(0,0){\circle{1.3\CLength}}}%
%    \makebox(0,1.0\CLength){\put(-.5\wd\CBox,0){#1}}}}

\def\circledlong#1{\sbox\CBox{#1}%
  \ifdim\wd\CBox>\ht\CBox \CLength=\wd\CBox\else\CLength=\ht\CBox\fi
    \makebox[1.2\CLength]{\makebox(0,0.6\CLength){\put(0,0){\circle{1.3\CLength}}}%
    \makebox(0,0.6\CLength){\put(-.5\wd\CBox,0){#1}}}}

\lstset{
	basicstyle=\footnotesize\ttfamily,
	breaklines=true,
    frame=tb, % draw a frame at the top and bottom of the code block
    tabsize=4, % tab space width
    showstringspaces=false, % don't mark spaces in strings
    numbers=left, % display line numbers on the left
    commentstyle=\color{Red}, % comment color
    keywordstyle=\color{blue}, % keyword color
    stringstyle=\color{OliveGreen}, % string color
	xleftmargin=.25in %align numbers to left side
}

%------------------------------------------------
% This block contains THESIS COMMANDS
%------------------------------------------------

%%%%%%%%%%% APPROACH
\newcommand{\codegen}{\textit{do$_{code}$}\xspace}
\newcommand{\ct}{\textit{c\&t}\xspace}
\newcommand{\nlms}{NCMs\xspace}
\newcommand{\nlm}{NCM\xspace}

\newcommand{\lambdacodegen}{{$causal$Code$Gen$}\xspace}
\newcommand{\rhocodegen}{{$\rho$Code$Gen$}\xspace}
\newcommand{\ccp}{\textit{CCP}\xspace}
\newcommand{\js}{$JSD$\xspace}

\newcommand{\asofte}{\textit{A-Soft-E}\xspace}
\newcommand{\asw}{\textit{Software$_{2.0}$}\xspace}


\newcommand{\codeSeqRational}{{\textbf{\textit{codeSeqRational}}}\xspace}
\newcommand{\shapleyCode}{{\textbf{\textit{shapCode}}}\xspace}
\newcommand{\codeXplainer}{{\textbf{\textit{codeXplainer}}}\xspace}

%%%%%%%%%% Counterfactual Interventions
\newcommand{\datainterI}{\textit{ProgramRepair}\xspace}
\newcommand{\datainterII}{\textit{SemanticPreserving}\xspace}
\newcommand{\datainterIII}{\textit{UnCommenting}\xspace}

\newcommand{\modelinterI}{\textit{NumberLayers}\xspace}
\newcommand{\modelinterII}{\textit{NumberUnits}\xspace}

%%%%%%% Effects
\newcommand{\assoJS}{JS Dist.\xspace}
\newcommand{\assoPR}{Pearson\xspace}

%%%%%%%% Refutations

\newcommand{\rfi}{$\mathcal{R}_1$\xspace}

%%%%% DATASETS
\newcommand{\training}{\textit{CodeSearchNet}\xspace}
\newcommand{\BuggyTB}{\textit{BuggyTB}\xspace}
\newcommand{\CommentsTB}{\textit{CommentsTB}\xspace}
\newcommand{\BigCloneIITB}{\textit{BigClone2TB}\xspace}
\newcommand{\BigCloneIIITB}{\textit{BigClone3TB}\xspace}
\newcommand{\BigCloneTB}{\textit{BigCloneTB}\xspace}

%%%%%% TAXONOMY
\newcommand{\blocks}{\texttt{\small[blocks]}\xspace}
\newcommand{\tests}{\texttt{\small[tests]}\xspace}
\newcommand{\oop}{\texttt{\small[oop]}\xspace}
\newcommand{\declarations}{\texttt{\small[declarations]}\xspace}
\newcommand{\exceptions}{\texttt{\small[exceptions]}\xspace}
\newcommand{\datatype}{\texttt{\small[datatype]}\xspace}
\newcommand{\loops}{\texttt{\small[loops]}\xspace}
\newcommand{\operators}{\texttt{\small[operators]}\xspace}
\newcommand{\conditionals}{\texttt{\small[conditionals]}\xspace}
\newcommand{\extra}{\texttt{\small[extraTokens]}\xspace}

%%%%%%%% REFERENCES
\newcommand{\secref}[1]{Sec.\S~\ref{#1}\xspace}
\newcommand{\chapref}[1]{Chapter~\ref{#1}\xspace}
\newcommand{\appref}[1]{Appendix~\ref{#1}\xspace}
\newcommand{\figref}[1]{Fig.~\ref{#1}\xspace}
\newcommand{\listref}[1]{Listing~\ref{#1}\xspace}
\newcommand{\equaref}[1]{Equation~\ref{#1}\xspace}
\newcommand{\tabref}[1]{Table~\ref{#1}\xspace}

%%%%%%%%MODELS
\newcommand{\rnn}{RNN$_{1,1024}$\xspace}
\newcommand{\gru}{GRU$_{1,1024}$\xspace}
\newcommand{\grui}{GRU$_{2,1024}$\xspace}
\newcommand{\gruii}{GRU$_{3,1024}$\xspace}
\newcommand{\gruiii}{GRU$_{1,512}$\xspace}
\newcommand{\gruiv}{GRU$_{1,2048}$\xspace}

\newcommand{\tf}{TF$_{6,12}$\xspace}
\newcommand{\tfi}{TF$_{12,12}$\xspace}
\newcommand{\tfii}{TF$_{24,12}$\xspace}

\newcommand{\Comet}{{\sc Comet}\xspace}
\newcommand{\Comets}{{\sc Comet's}\xspace}

%%%%% COMMONS
%\newcommand{\ie}{\textit{i.e.,}\xspace}
%\newcommand{\eg}{\textit{e.g.,}\xspace}
\newcommand{\hairsp}{\hspace{1pt}} % Command to print a very short space
\newcommand{\ie}{\textit{i.\hairsp{}e.}\xspace} % Command to print i.e.
\newcommand{\eg}{\textit{e.\hairsp{}g.}\xspace} % Command to print e.g.
\newcommand{\etc}{\textit{etc.}\xspace}
\newcommand{\etal}{et al.\xspace}
\newcommand{\etals}{et al.'s\xspace}
\newcommand{\aka}{\textit{a.k.a.}\xspace}	
\newcommand{\REF}{{\color{red} \textbf{[REFS]}}\xspace}

%%%%% THEOREM

\newtheorem{definition}{Definition}[chapter]
\newtheorem{exmp}{\underline{Example \textbf{. ---}} }[chapter]

%%%%% FIGURES
\newcommand*\circled[1]{\tikz[baseline=(char.base)]{
            \node[shape=circle,draw,inner sep=0.5pt] (char) {#1};}}


%%%%% COMMENTS
\newboolean{showcomments} %comments
\setboolean{showcomments}{true} %comments
\ifthenelse{\boolean{showcomments}} %comments
  {\newcommand{\nb}[2]{
    \fbox{\bfseries\sffamily\scriptsize#1}
    {\sf\small$\blacktriangleright$\textit{#2}$\blacktriangleleft$}
   }
   \newcommand{\cvsversion}{\emph{\scriptsize$-$Id: macro.tex,v 1.9 2005/12/09 22:38:33 giulio Exp $}}
  }
  {\newcommand{\nb}[2]{}
   \newcommand{\cvsversion}{}
  }

\newcommand{\david}[1]{ {\color{blue} \nb{DAVID}{#1} } } %personalization
\newcommand{\DENYS}[1]{{\color{blue} \nb{DENYS}{#1}}}   %personalization
%\newcommand{\comment}[1]{}


%%%%% TUFTE

\newcommand\newthought[1]{%
   \addvspace{1.0\baselineskip plus 0.5ex minus 0.2ex}%
   \noindent\textsc{#1}%
}

\newtcolorbox{boxK}{
    fontupper = \small,
    sharpish corners, % better drop shadow
    boxrule = 0pt,
    toprule = 4.5pt, % top rule weight
    enhanced,
    fuzzy shadow = {0pt}{-2pt}{-0.5pt}{0.5pt}{black!35} % {xshift}{yshift}{offset}{step}{options} 
}

\newcommand{\marginnote}[1]{ \begin{boxK} #1 \end{boxK} } %Adapting MarginNote for Old WM Template


% Margin figure environment

\newenvironment{marginfigure}[1][h]%
  {\begin{figure}[#1] }
  {\end{figure}}


%%
% Margin table environment

\newenvironment{margintable}[1][h]%
  {\begin{table}[#1]}
  {\end{table}}
