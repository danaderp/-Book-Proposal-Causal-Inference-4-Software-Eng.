%  W&M PhD Dissertation LaTeX File
\documentclass[cpp,11pt]{wmthesis}
\usepackage[linesnumbered, ruled, vlined]{algorithm2e}
\usepackage{graphicx}
\usepackage{balance}
\usepackage{caption}
\usepackage{moresize}
\usepackage{pbox}
\usepackage{mathtools}
\usepackage[TABBOTCAP]{subfigure}
\setlength{\textfloatsep}{2pt}
\usepackage{paralist}
\usepackage{hyperref}
\usepackage[T1]{fontenc}
\usepackage{balance}
\usepackage[dvipsnames,table,xcdraw]{xcolor}
\usepackage{multirow}
\usepackage{multicol}
\usepackage{amsmath}
\usepackage{listings}
\usepackage{setspace}
\usepackage{verbatim}
\usepackage[all]{nowidow}
\usepackage{float}
\usepackage{xspace}
\usepackage{amssymb}
\usepackage{ifthen}
\usepackage{pifont}
\usepackage{textcomp}
\usepackage{pict2e,picture}
\usepackage{url}
%\usepackage{pstricks}
\usepackage{epsfig}
\usepackage[subfigure]{tocloft}
\usepackage{amsfonts}
\usepackage{amssymb}
\usepackage{latexsym}
\usepackage{booktabs}
\usepackage{tabularx}
\usepackage{rotating}
\usepackage[numbers]{natbib}
\usepackage{mdwlist}
\usepackage{colortbl}
\usepackage[normalem]{ulem}
\renewcommand{\rmdefault}{cmr} % Arial
\renewcommand{\sfdefault}{cmr} % Arial
\renewcommand{\cftchapfont}{\rm } %no bold in toc
\renewcommand{\cftchappagefont}{\rm } %no bold in toc
\usepackage{tikz}

%----------------------------------------------------------------------------------------
%	MACROS
%----------------------------------------------------------------------------------------
\makeatletter

% this  is a easy way to add and highlight new text  ...
% just comment in/out the \tnew macro ..

\newcommand{\tnew}[1]{{\bf { #1 }} }
%\newcommand{\tnew}[1]{{ { #1 }} }

% math and theorem definition

\newcommand{\ndef}{\stackrel{\rm def}{=}}

% this is used for draft only

%\renewcommand{\baselinestretch}{2}

% just to number pages in the draft

\useunder{\uline}{\ul}{}

% nothing i.e., no-numbering final and camera ready

%\pagestyle{empty}

\newcommand{\CrashScopesp}{{\textsc CrashScope~}}
\newcommand{\CrashScopes}{{\textsc \small CrashScope's~}}
\newcommand{\CrashDroid}{{\textsc CrashDroid~}}
\newcommand{\CrashScopebf}{{ \textbf{\textsc CrashScope~}}}

\newcommand{\ReDraw}{{\textsc{\small ReDraw}}\xspace}
\newcommand{\ReDraws}{{\textsc{\small ReDraw's}}\xspace}
\newcommand{\Remaui}{{\textsc{\small Remaui}}\xspace}
\newcommand{\Remauis}{{\textsc{\small Remaui's}}\xspace}
\newcommand{\CrashScope}{{\textsc{\small CrashScope}}\xspace}
\newcommand{\MonkeyLab}{{\textsc{\small MonkeyLab}}\xspace}
\newcommand{\pixcode}{{pix2code}\xspace}

\newcommand{\GVTsp}{{\textsc{\small Gvt~}}}
\newcommand{\GVT}{{\textsc{\small Gvt}}}
\newcommand{\GVTs}{{\textsc{\small Gvt's~}}}
\newcommand{\dv}{{\textit{DV}}\xspace}
\newcommand{\dvs}{{\textit{DVs}}\xspace}
\newcommand{\gc}{{\textit{GC}}\xspace}
\newcommand{\gcs}{{\textit{GCs}}\xspace}
\newcommand{\mgc}{{\textit{M-GC}}\xspace}
\newcommand{\mgcs}{{\textit{M-GCs}}\xspace}
\newcommand{\igc}{{\textit{I-GC}}\xspace}
\newcommand{\igcs}{{\textit{I-GCs}}\xspace}


\newcommand{\ReDrawAbs}{{R}{\ssmall E}{D}{\ssmall RAW}\xspace}




\newsavebox\CBox
\newlength\CLength
%\def\circled#1{\sbox\CBox{#1}%
%  \ifdim\wd\CBox>\ht\CBox \CLength=\wd\CBox\else\CLength=\ht\CBox\fi
%    \makebox[1.2\CLength]{\makebox(0,0.9\CLength){\put(0,0){\circle{1.3\CLength}}}%
%    \makebox(0,1.0\CLength){\put(-.5\wd\CBox,0){#1}}}}

\def\circledlong#1{\sbox\CBox{#1}%
  \ifdim\wd\CBox>\ht\CBox \CLength=\wd\CBox\else\CLength=\ht\CBox\fi
    \makebox[1.2\CLength]{\makebox(0,0.6\CLength){\put(0,0){\circle{1.3\CLength}}}%
    \makebox(0,0.6\CLength){\put(-.5\wd\CBox,0){#1}}}}

\lstset{
	basicstyle=\footnotesize\ttfamily,
	breaklines=true,
    frame=tb, % draw a frame at the top and bottom of the code block
    tabsize=4, % tab space width
    showstringspaces=false, % don't mark spaces in strings
    numbers=left, % display line numbers on the left
    commentstyle=\color{Red}, % comment color
    keywordstyle=\color{blue}, % keyword color
    stringstyle=\color{OliveGreen}, % string color
	xleftmargin=.25in %align numbers to left side
}

%------------------------------------------------
% This block contains THESIS COMMANDS
%------------------------------------------------

%%%%%%%%%%% APPROACH
\newcommand{\codegen}{\textit{do$_{code}$}\xspace}
\newcommand{\ct}{\textit{c\&t}\xspace}
\newcommand{\nlms}{NCMs\xspace}
\newcommand{\nlm}{NCM\xspace}

\newcommand{\lambdacodegen}{{$causal$Code$Gen$}\xspace}
\newcommand{\rhocodegen}{{$\rho$Code$Gen$}\xspace}
\newcommand{\ccp}{\textit{CCP}\xspace}
\newcommand{\js}{$JSD$\xspace}

\newcommand{\asofte}{\textit{A-Soft-E}\xspace}
\newcommand{\asw}{\textit{Software$_{2.0}$}\xspace}


\newcommand{\codeSeqRational}{{\textbf{\textit{codeSeqRational}}}\xspace}
\newcommand{\shapleyCode}{{\textbf{\textit{shapCode}}}\xspace}
\newcommand{\codeXplainer}{{\textbf{\textit{codeXplainer}}}\xspace}

%%%%%%%%%% Counterfactual Interventions
\newcommand{\datainterI}{\textit{ProgramRepair}\xspace}
\newcommand{\datainterII}{\textit{SemanticPreserving}\xspace}
\newcommand{\datainterIII}{\textit{UnCommenting}\xspace}

\newcommand{\modelinterI}{\textit{NumberLayers}\xspace}
\newcommand{\modelinterII}{\textit{NumberUnits}\xspace}

%%%%%%% Effects
\newcommand{\assoJS}{JS Dist.\xspace}
\newcommand{\assoPR}{Pearson\xspace}

%%%%%%%% Refutations

\newcommand{\rfi}{$\mathcal{R}_1$\xspace}

%%%%% DATASETS
\newcommand{\training}{\textit{CodeSearchNet}\xspace}
\newcommand{\BuggyTB}{\textit{BuggyTB}\xspace}
\newcommand{\CommentsTB}{\textit{CommentsTB}\xspace}
\newcommand{\BigCloneIITB}{\textit{BigClone2TB}\xspace}
\newcommand{\BigCloneIIITB}{\textit{BigClone3TB}\xspace}
\newcommand{\BigCloneTB}{\textit{BigCloneTB}\xspace}

%%%%%% TAXONOMY
\newcommand{\blocks}{\texttt{\small[blocks]}\xspace}
\newcommand{\tests}{\texttt{\small[tests]}\xspace}
\newcommand{\oop}{\texttt{\small[oop]}\xspace}
\newcommand{\declarations}{\texttt{\small[declarations]}\xspace}
\newcommand{\exceptions}{\texttt{\small[exceptions]}\xspace}
\newcommand{\datatype}{\texttt{\small[datatype]}\xspace}
\newcommand{\loops}{\texttt{\small[loops]}\xspace}
\newcommand{\operators}{\texttt{\small[operators]}\xspace}
\newcommand{\conditionals}{\texttt{\small[conditionals]}\xspace}
\newcommand{\extra}{\texttt{\small[extraTokens]}\xspace}

%%%%%%%% REFERENCES
\newcommand{\secref}[1]{Sec.\S~\ref{#1}\xspace}
\newcommand{\chapref}[1]{Chapter~\ref{#1}\xspace}
\newcommand{\appref}[1]{Appendix~\ref{#1}\xspace}
\newcommand{\figref}[1]{Fig.~\ref{#1}\xspace}
\newcommand{\listref}[1]{Listing~\ref{#1}\xspace}
\newcommand{\equaref}[1]{Equation~\ref{#1}\xspace}
\newcommand{\tabref}[1]{Table~\ref{#1}\xspace}

%%%%%%%%MODELS
\newcommand{\rnn}{RNN$_{1,1024}$\xspace}
\newcommand{\gru}{GRU$_{1,1024}$\xspace}
\newcommand{\grui}{GRU$_{2,1024}$\xspace}
\newcommand{\gruii}{GRU$_{3,1024}$\xspace}
\newcommand{\gruiii}{GRU$_{1,512}$\xspace}
\newcommand{\gruiv}{GRU$_{1,2048}$\xspace}

\newcommand{\tf}{TF$_{6,12}$\xspace}
\newcommand{\tfi}{TF$_{12,12}$\xspace}
\newcommand{\tfii}{TF$_{24,12}$\xspace}

\newcommand{\Comet}{{\sc Comet}\xspace}
\newcommand{\Comets}{{\sc Comet's}\xspace}

%%%%% COMMONS
%\newcommand{\ie}{\textit{i.e.,}\xspace}
%\newcommand{\eg}{\textit{e.g.,}\xspace}
\newcommand{\hairsp}{\hspace{1pt}} % Command to print a very short space
\newcommand{\ie}{\textit{i.\hairsp{}e.}\xspace} % Command to print i.e.
\newcommand{\eg}{\textit{e.\hairsp{}g.}\xspace} % Command to print e.g.
\newcommand{\etc}{\textit{etc.}\xspace}
\newcommand{\etal}{et al.\xspace}
\newcommand{\etals}{et al.'s\xspace}
\newcommand{\aka}{\textit{a.k.a.}\xspace}	
\newcommand{\REF}{{\color{red} \textbf{[REFS]}}\xspace}

%%%%% THEOREM
\newtheorem{definition}{Definition}[chapter]
\newtheorem{exmp}{\underline{Example}}[chapter]

%%%%% FIGURES
\newcommand*\circled[1]{\tikz[baseline=(char.base)]{
            \node[shape=circle,draw,inner sep=0.5pt] (char) {#1};}}


%%%%% COMMENTS
\newboolean{showcomments} %comments
\setboolean{showcomments}{true} %comments
\ifthenelse{\boolean{showcomments}} %comments
  {\newcommand{\nb}[2]{
    \fbox{\bfseries\sffamily\scriptsize#1}
    {\sf\small$\blacktriangleright$\textit{#2}$\blacktriangleleft$}
   }
   \newcommand{\cvsversion}{\emph{\scriptsize$-$Id: macro.tex,v 1.9 2005/12/09 22:38:33 giulio Exp $}}
  }
  {\newcommand{\nb}[2]{}
   \newcommand{\cvsversion}{}
  }

\newcommand{\david}[1]{ {\color{blue} \nb{DAVID}{#1} } } %personalization
\newcommand{\DENYS}[1]{{\color{blue} \nb{DENYS}{#1}}}   %personalization
%\newcommand{\comment}[1]{}


%%%%% TUFTE

\newcommand\newthought[1]{%
   \addvspace{1.0\baselineskip plus 0.5ex minus 0.2ex}%
   \noindent\textsc{#1}%
}




% The wmthesis class is based on the latex report class whic
% only indents paragraphs if they immediately follow other paragraphs.  The
% dissertation lady says this is wrong.  I tend to give more credence
% to Dr. Knuth (author of TeX) on this issue, since the other way looks really
% crappy.  If you want the first line of every paragraph indented,
% uncomment the next line to include the indentfirst package. -- rem
% \usepackage{indentfirst}
% Not sure if this is still an option -- Ruth

\def\BEGINITEMIZE{\begin{itemize}}
\def\ENDITEMIZE{\end{itemize}}

\def\defaultpenalty{1000} \clubpenalty=\defaultpenalty
\widowpenalty=\defaultpenalty

%%%%%%%%%%%%%%%%%%%%%%%%%%%%%%%%%%%
%%   I put all of my specially defined commands in this file to ensure
%%   that I maintain consistency since I changed some notation 
%%   usage between publications. Feel free to delete and/or modify
%%   to suit your purpose.
%%%%%%%%%%%%%%%%%%%%%%%%%%%%%%%%%%%
%%--Some general purpose macros.
%%----------------------------------------------------------------
% Usage:
%     \inputfig{filename} 
%        or
%     \inputfig[scaling_factor]{filename} 
%        filename.ps    is expected to be in the directory ../figs
%        scaling_factor the amount of scaling to be applied 
%                       (a decimal fraction between 0.0 and 1.0
%           
\newcommand{\inputfig}[2][\empty]{ %
   \begin{center} %
   	\ifx\empty#1 \includegraphics{../figs/#2}
	\else\scalebox{#1}{\includegraphics{../figs/#2}}\fi
   \end{center}}

% Usage: 
%     \inputplot{filename}
%        filename.ps is expected to be in the directory ../plots/ps
\newcommand{\inputplot}[1]{
   \begin{center}\includegraphics{../plots/ps/#1.ps}\end{center}}

%%----------------------------------------------------------------
% Set the title that will be printed on the Contents page
%%----------------------------------------------------------------
% The negative vspace is used to make sure that only one line is
% between the title and the first line for each of these pages.
\renewcommand{\contentsname}{\begin{center}\Large\normalfont TABLE OF CONTENTS\vspace{-.5in}\end{center}}
\renewcommand\listfigurename{\begin{center}\Large\normalfont LIST OF FIGURES\vspace{-.35in}\end{center}}
\renewcommand\listtablename{\begin{center}\Large\normalfont LIST OF TABLES\vspace{-.35in}\end{center}}

%%----------------------------------------------------------------
%%----------------------------------------------------------------

\begin{document}
\doublespacing

%----------------------------------------------------------------------------------------
%	BOOK META-INFORMATION
%----------------------------------------------------------------------------------------


%%--Set thesis info.
%%--*IMPORTANT* Title cannot be in ALL CAPS
\thesisTitle{On Exploring Synergies at the Intersection of Causality and Artificial Intelligence  to Automate Software Engineering Tasks}

\thesisAuthor[David A. Nader Palacio]{David N. Palacio}

\thesisMonth{November}

\thesisYear{2022}

\thesisAdvisor{Associate Professor Denys Poshyvanyk}

% location and degrees added 
% note that the degree should be spelled out, not abbreviated
\thesisLocation{Hometown, HomeState, HomeCountry}
\thesisDegreeOne{Bachelor of Engineering, National University of Colombia, 2012}
\thesisDegreeTwo{Master of Engineering, National University of Colombia, 2017}
%\thesisDegreeThree{Master of Science, College of William and Mary, 2015}
\thesisCommittee[Computer Science]{\ThesisAdvisor}
\thesisCommittee[Computer Science]{Associate Professor FirstName LastName}
\thesisCommittee[Computer Science]{Assistant Professor FirstName LastName}
\thesisCommittee[Computer Science]{Assistant Professor FirstName LastName}
\thesisCommittee[Computer Science]{Professor FirstName LastName}

%----------------------------------------------------------------------------------------
%	DEDICATION PAGE
%----------------------------------------------------------------------------------------
\thesisDedication{Insert heartfelt dedication here...}

%%-- Insert contents of acknowledge.tex and abstract.tex.  Don't
%%forget to check these files for formatting hints.

%----------------------------------------------------------------------------------------
%	ACK PAGE
%----------------------------------------------------------------------------------------
\thesisAcknowledge{wm_old_template/acknowledge}

%----------------------------------------------------------------------------------------
%	ABSTRACT PAGE
%----------------------------------------------------------------------------------------

\thesisAbstract{wm_old_template/abstract} 

%%--Create the dissertation Prolog
\makeThesisProlog

%%--Include the dissertation chapters.

%----------------------------------------------------------------------------------------
%	CHAPTER 1: INTRODUCTION
%----------------------------------------------------------------------------------------

\chapter{Introduction} % The asterisk * leaves out this chapter from the table of contents
\label{ch:intro}


Software engineering (SE) research investigates questions pertaining to the design, development, maintenance, testing, and evolution of software systems. As software continues to pervade a wide range of industries, both open- and closed-source code repositories have grown to become unprecedentedly large and complex. This has resulted in an increase of unstructured, unlabeled, yet important data including requirements, design documents, source code files, test cases, and defect reports. Previously, the software engineering community has applied canonical machine learning (ML) techniques to identify patterns and unique relationships within this data to automate or enhance many tasks typically performed manually by developers. Unfortunately, the process of implementing ML techniques can be a tedious exercise in careful feature engineering, wherein researchers experiment with identifying salient attributes of data that can be leveraged to help solve a given problem or automate a given task.

However, with recent improvements in computational power and the amount of memory available in modern computer architectures, an advancement to traditional ML approaches has arisen called Deep Learning (DL). Deep learning represents a fundamental shift in the manner by which machines learn patterns from data by \textit{automatically} extracting salient features for a given computational task as opposed to relying upon human intuition. Deep Learning approaches are characterized by architectures comprised of several layers that perform mathematical transformations on data passing through them. These transformations are controlled by sets of learnable parameters that are adjusted using a variety of learning and optimization algorithms. These computational layers and parameters form models that can be trained for specific tasks by updating the parameters according to a model's performance on a set of training data. Given the immense amount of structured and unstructured data in software repositories that are likely to contain hidden patterns, DL techniques have ushered in advancements across a range of tasks in software engineering research including automatic program repair~\citep{Tufano2018}, code suggestion~\citep{Gu2018}, defect prediction~\citep{Wang2016}, malware detection \cite{Li2018}, feature location~\citep{Corley2015}, among many others~\citep{Ma2018, Wan2018, Liu2018, White2016, Xu2016, Guo2017, Tian2018a, Liu2017}. A recent report from the 2019 NSF Workshop on Deep Leaning \& Software Engineering has referred to this area of work as Deep Learning for Software Engineering (DL4SE)~\citep{dlse19-report}. 

The applications of DL to improve and automate SE tasks points to a clear synergy between ongoing research in SE and DL. However, in order to effectively chart the most impactful path forward for research at the intersection of these two fields, researchers need a clear map of what has been done, what has been successful, and what can be improved. %
In an effort to map and guide research at the intersection of DL and SE, we conducted a systematic literature review (SLR) to identify and systematically enumerate the synergies between the two research fields. As a result of the analysis performed in our SLR, we synthesize a detailed \textit{research roadmap} of past work on DL techniques applied to SE tasks\footnote{It should be noted that another area, known as Software Engineering for Deep Learning (SE4DL), which explores improvements to engineering processes for DL-based systems, was also identified at the 2019 NSF workshop. However, the number of papers we identified on this topic was small, and mostly centered around emerging testing techniques for DL models. Therefore, we reserve a survey on this line of research for future work.} (\ie DL4SE), complete with identified open challenges and best practices for applying DL techniques to SE-related tasks and data. Additionally, we analyzed the impacts of these DL-based approaches and discuss some observed concerns related to the potential reproducibility and replicability of our studied body of literature. %

%%%%%%%%%%%%%%%%%%%%%%%%%%%%%%%%%%%%%%%%%%

% Paragraph1: Motivation
\newthought{This dissertation} explores the usage of a mathematical structure that assess the causal effect of automation process in the context of Software Engineering. Such mathematical structure embodies the causal calculus to perform estimations of software variables affecting other variables. However, the software variables under analysis are not coming from the classical perspective of software engineering but from the field where Software Engineering is generated by Artificial Intelligence mechanism. This field is introduced as \textit{Artificial Software Engineering} (\asofte). In order to \textit{Artificial Software Engineering} achieve understandability (or interpretability in a Machine Learning context). It is required to adapt, formalize, and evaluate Causal Inference concepts or causal mathematical structures that help aid to uncover causal effects. Therefore, we need a causal artificial software structure at the interface of causality and artificial software engineering.  

% Paragraph2: What is the specific problem?
Although Causal Calculus has been introduced since the 80s, there no exist a formalization of a causal artificial software structure...  

% Paragraph3: What is the main contribution of the dissertation?
This dissertation poses a causal structure for the problem of deep code retrieval and deep code interpretability...

% Paragraph4: Differences of what I am doing and others have done

% Paragraph5: The structure of the dissertation.





%------------------------------------------------
\section{A Motivating Example}

%------------------------------------------------
\section{Terminology}

\subsection{Interpretability}
Neural Language Models (NLM) are increasingly being used in Software Engineering as \textit{Code Generators} showing promising results in generating correct and realistic code. Intepretability represents one of the major challenge limiting the deployment and usage of these models in practice, since the causal relationship between the input and output is often unclear and no cues are provided informing what influenced the generation of a specific snippet of code. In this patent we propose \codeSeqRational, a framework that allows to extract practical interpretability insights for NLM-based systems for code-related tasks. Our approach is based on a greedy algorithm which extracts the smallest subset of tokens (rationales) from the input sufficient to predict each token in the output. Next, these rationales are mapped into human-interpretable concepts by a set of mapping functions, which assign tokens to a set of categories. These include code-specific categories (\ie structural and identifiers extracted with a code parser), as well as natural language categories (\eg verbs and nouns extracted with a NL context-free grammar parser). Finally, tokens are grouped into location-aware scopes. This infrastructure allows researchers and practitioners to debug NLM outputs as well as further optimize the input to these models evaluating the importance of each token, category, or scope.

%----------------------------------------------------------------------------------------
%	CHAPTER 2: PRELIMINARIES
%----------------------------------------------------------------------------------------

\chapter{Preliminaries}
\label{ch:preliminaries}



%------------------------------------------------

\section{From \sw to \asw}
\label{sec:asofte}

Software engineering (SE) research investigates questions pertaining to the design, development, maintenance, testing, and evolution of software systems. As software continues to pervade a wide range of industries, both open- and closed-source code repositories have grown to become unprecedentedly large and complex. This has resulted in an increase of unstructured, unlabeled, yet important data including requirements, design documents, source code files, test cases, and defect reports. Previously, the software engineering community has applied canonical Artificial Intelligence (AI) techniques to identify patterns and unique relationships within this data to automate or enhance many tasks typically performed manually by developers. Unfortunately, the process of implementing canonical AI techniques can be a tedious exercise in careful feature engineering, wherein researchers experiment with identifying salient attributes of data that can be leveraged to help solve a given problem or automate a given task.

However, with recent improvements in computational power and the amount of memory available in modern computer architectures, an advancement to traditional AI approaches has arisen called Deep Learning (DL). Deep learning represents a fundamental shift in the manner by which machines learn patterns from data by \textit{automatically} extracting salient features for a given computational task as opposed to relying upon human intuition. Deep Learning approaches are characterized by architectures comprised of several layers that perform mathematical transformations on data passing through them. These transformations are controlled by sets of learnable parameters that are adjusted using a variety of learning and optimization algorithms. These computational layers and parameters form models that can be trained for specific tasks by updating the parameters according to a model's performance on a set of training data. Given the immense amount of structured and unstructured data in software repositories that are likely to contain hidden patterns, DL techniques have ushered in advancements across a range of tasks in software engineering research including automatic program repair~\citep{Tufano2018}, code suggestion~\citep{Gu2018}, defect prediction~\citep{Wang2016}, malware detection \cite{Li2018}, feature location~\citep{Corley2015}, among many others~\citep{Ma2018, Wan2018, Liu2018, White2016, Xu2016, Guo2017, Tian2018a, Liu2017}. A recent report from the 2019 NSF Workshop on Deep Leaning \& Software Engineering has referred to this area of work as Deep Learning for Software Engineering (\dlse)~\citep{dlse19-report}. 

The applications of DL to improve and automate SE tasks points to a clear synergy between ongoing research in SE and DL. However, in order to effectively chart the most impactful path forward for research at the intersection of these two fields, researchers need a clear map of what has been done, what has been successful, and what can be improved. %
In an effort to map and guide research at the intersection of DL and SE, we conducted a systematic literature review (SLR) to identify and systematically enumerate the synergies between the two research fields. As a result of the analysis performed in our SLR, we synthesize a detailed \textit{research roadmap} of past work on DL techniques applied to SE tasks\footnote{It should be noted that another area, known as Software Engineering for Deep Learning (SE4DL), which explores improvements to engineering processes for DL-based systems, was also identified at the 2019 NSF workshop. However, the number of papers we identified on this topic was small, and mostly centered around emerging testing techniques for DL models. Therefore, we reserve a survey on this line of research for future work.} (\ie \dlse), complete with identified open challenges and best practices for applying DL techniques to SE-related tasks and data. Additionally, we analyzed the impacts of these DL-based approaches and discuss some observed concerns related to the potential reproducibility and replicability of our studied body of literature. % 

%------------------------------------------------

\section{Neural Code Generators}
\label{sec:ncg}

\textbf{Code Representation.} Assuming a train corpus of code data (\eg code snippets) $x \in X$ is represented by a distribution with a form $p_{data}(X)$. A Neural Code Generator (\ncg) is a probability distribution $p_{model}(X)$ \textit{statistically learned} by an autoregressive (\ie Transformer) or recursive (\ie RNN) deep learning model. The question \ncg's attempt to answer is \textit{How can we learn $p_{model}$ similar to $p_{data}$?} Both are joint distributions but $p_{data}$ and $p_{model}$ behave distinctly. On the one hand, we consider $p_{data}$ observational since we are just using samples from code snippets written by humans. On the other hand, $p_{model}$ is a reconstructed probability from generalizing human snippets. 

We use deep generative theory, statistical analysis, information theory, and causal inference to compare machine with human code samples using \textit{interpretability techniques}. The distribution $p_{model}$ can be sampled in two ways: conditioned $p(x|w_{<t},\theta)$ and unconditioned  $p(x|w_0,\theta)$. Note that the unconditioned distribution is approximated by performing semi-supervised learning conditioning on hyperparameters $\theta$ and a special \textit{starting token} $w_0$. If we want to represent a specific deep learning approach generating a sequence of tokens, then it is employed the notation $p_{model}(x|w_0, \theta)$, where the $model$ is any DL architecture. 

Therefore, $p_{data} \approx p_{model}(x|w_0, \theta)$ is an approximation of human generated code. We can condition based on the model parameters $\theta$ and sub-tokens of the code corpora $w$. The learning parameters $\theta$ refers to the variables that affect the generation. Then, the parameters for representing the distribution (i.e. Mixture Models, embeddings, or PCA) are not taken into consideration for the generative process. We obtain a conditional distribution $p(x|w,\theta)$. We say that $p(x|w,\theta)$ is observational because we are estimating $X$ given that we \textbf{observe} variable $w$ takes value $w_{<t}$ and $\Theta$ takes value $\theta$. In this particular case, $w$ is self-contained in the data since $w \subseteq X$ and $\theta$ is assumed to be contain in the train data. In any case, we are passively observing features from a human generated corpora. 

%% Unconditioned and Conditioned Concepts
Furthermore, Neural Code Generators (\ncg) can be split into \textit{\textbf{Unconditioned}} and \textit{\textbf{Conditioned}} models. Conditioned models the generation process receives a sequence as a starting point, thus following a \textit{completion} approach. While unconditioned Language Models (\ulm) follow an open-ended approach: the provided context corresponds to the special \textit{Beginning of Sentence} token.

\textbf{Unconditioned Sampling.} Holtzman et al \citep{Holtzman2019} and Nguyen \citep{Nguyen2021} noted that the \textit{decoding strategy} plays an important role in the generative process of language models (i.e., autoregressive generation). \textit{Decoding strategy} refers to the mechanism used for selecting the output token at each step of the generation process based on the autoregressive models \citep{Holtzman2019}.  At each timestep, the LM produces the probability of each word in the vocabulary being the likely next word to be selected. There are several maximization-based decoding methods such as beam or greedy search, that lead to degeneration of the produced text as noted by Holtzman et al \cite{Holtzman2019}. Holtzman\citep{Holtzman2019} and Nguyen \cite{Nguyen2021} also argued that it is common to leverage \textit{stochastic} decoding methods. These methods aim to introduce a degree of randomness to the generation process, giving the models less chance of repeating themselves. Two popular stochastic decoding methods are \textit{top-k sampling} and \textit{temperature-sampling}.

\textit{Top-k} sampling was originally introduced by Fan et al. \citep{Fan2018}. The authors aimed to train models able to produce coherent and fluent passages of text regarding a topic for the task of \textit{story generation}. The proposed sampling scheme consists of filtering  the $k$ most likely next words and next, redistributing the probability mass among only those $k$ next words. This strategy is sensitive to the choice of parameter $k$.

\textit{Temperature-sampling} refers to a mechanism to shape the distribution resulting from a softmax layer used to pick the next token at each generation step as noted by Ficler and Goldberg \citep{Ficler}. This process aims to increase the likelihood of high probability words and decrease the likelihood of low probability words. Such behavior is attained by modifying the so-called temperature parameter of the softmax function.

%------------------------------------------------

\section{Deep Code Retrieval Problem}
\label{sec:deep-retrieval-problem}

\david{introduce what software retrieval means. Rewrite the related work to give a proper introduction of the concept}

We focus our discussion of related work on prior techniques that have, in limited contexts, (i) considered novel or hybrid textual similarity measures, (ii) modeled the effects of multiple types of artifacts, or (iii) incorporated developer expertise. We then conclude with a statement distilling \Comets novelty.

\noindent{\textbf{Novel/Hybird Textual Similarity Measures:}} Guo \etal~\cite{Guo:ICSE'17} proposed an approach for candidate trace link prediction that uses a semantically enhanced similarity measure based on Deep Learning (DL) techniques. However, unlike \Comet, this technique requires pre-existing trace links in order to train the DL classifier.  In contrast, \Comet does not require known links for the projects it is applied to, but rather requires a project to serve as a tuning set. We show that \Comet performs well when tuned and tested on different datasets, outperforming Guo \etals DL-based approach when it is trained in a similar manner. Gethers \etal~\citep{Gethers:ICSM'11}, implemented an approach that is capable of combining information from canonical IR techniques (\ie VSM, Jensen-Shannon) with Topic Modeling techniques. However, their approach can only combine two IR/ML techniques, whereas \Comet can combine and leverage the observations from several IR/ML techniques, and combine this with other information such as expert feedback and transitive links. 

\noindent{\textbf{Modeling of Multiple Artifacts:}} Rath \etal~\citep{Rath:ICSE'18} recently explored linking nontraditional information including issues and commits, and Cleland-Huang \etal~\citep{Cleland-Huang:ICSE'10} have investigated linking regulatory codes to product level requirements.  \Comets model has the potential to improve trace link recovery in these scenarios both through its more robust modeling of textual similarity, and through incorporation of transitive link information. Furtado \etal~\cite{Furtado:RE'16}, explored traceability in the context of agile development, and Nishikawa \etal~\citep{Nishikawa:ICSME'15} first explored the use of transitive links in a deterministic traceability model. Additionally, Kuang \etal used the closeness of code dependencies, to help improve IR-based traceability recovery~\citep{Kuang:SANER'17}. However, none of these approaches is capable of incorporating transitive links while also considering combined textual similarity metrics and developer feedback.

\noindent{\textbf{Incorporation of Developer Expertise:}} De Lucia \etal \citep{DeLucia:ICSM'06} and Hayes \etal~\citep{Hayes:TSE'06} analyzed approaches that use relevance feedback to improve trace link recovery. However, these approaches are either tied to a particular type of model (such as TF-IDF~\citep{DeLucia:ICSM'06}), or require knowledge of the underlying model to function optimally. In contrast, \Comet implements a lightweight, likert-based feedback collection mechanism that we illustrate can improve link accuracy even when only a small amount of feedback is collected.
 
\noindent{\textbf{Summary of Advancement over Prior Work:}} \Comets features facilitate its application to projects without any pre-existing trace links, and as our evaluation illustrates, allow it to perform consistently well across datasets. \Comet is able to combine information from transitive links with both robust textual similarity measures and lightweight developer feedback for improved accuracy. While some aspects of \Comets approach have been considered in limited contexts in prior work -- such as developer feedback~\citep{DeLucia:ICSM'06,Hayes:TSE'06} and restricted combinations of IR/ML techniques~\citep{Gethers:ICSM'11} -- there has never been a framework capable of combining all these aspects in a holistic approach. Our evaluation illustrates that \Comets holistic HBN is able to outperform baseline techniques on average.

\subsection{Formalization}

Our goal is to design a model that captures meaningful information regarding logical relationships between software artifacts, and then use this model to infer a set of candidate trace links. More specifically, given a set of source artifacts $S$ (\eg requirements, use cases) such that $S = \{S_{1},S_{2},\ldots S_{n}\}$ and a set of target artifacts $T$ (\eg source code files, test cases) such that $T = \{T_{1},T_{2},\ldots T_{n}\}$, we aim to infer whether a trace link $L$ exists between all possible pairs of artifacts in $S$ and $T$ such that $L = \{(s,t) | s\in S, t\in T, s\leftrightarrow t\}$ where each pair of artifacts $s$ and $t$ are said to be logical trace links.


\subsection{The Probabilistic Nature of Software Traceability}
\david{The probabilistic nature of software retrieval!}
The process of building software is not inherently deterministic, and is instead the result of decisions made by engineers over prolonged periods of time that may be hard to predict. Developer decisions related to nearly every observable phenomenon in modern software development are influenced by a combination of multiple factors. For instance, the presence of a functional bug may be influenced by the quality of related requirements, implementation constraints imposed by a given programming language~\citep{Ray:CACM'17}, or the change-proneness of underlying APIs~\citep{Linares-Vasquez:FSE'13}. Given that such factors are often hard to predict, there is a clear sense of randomness inherent to the software development process. Similarly, the existence of trace links among software artifacts is also likely to be influenced by several different effectively \textit{random} factors.  

These factors could include textual similarities between artifacts, programmatic associations between pieces of code, or even abstract notions of similarity held by expert developers.  For example, the textual quality of requirements or identifiers in code are typically a function of several factors such as the fluency and writing style of the author and the familiarity of key phrases chosen for identifiers \citep{Dasgupta:ICSME'13}. This may lead to variable names that may be perfectly clear to one engineer being indecipherable to another.  \textit{From this view point, the existence of trace links between software artifacts can be thought of as an inherently a probabilistic phenomenon}.

\subsection{Traceability as a Bayesian Inference Problem}

Hence, in order to effectively model trace links among software artifacts, it is necessary to model a collection of random factors that influence \textit{the probability that a trace link exists}. Thus, the process of deriving trace links can be modeled as a \textit{Bayesian inference} problem, wherein a probability distribution representing the existence of a trace link between two artifacts can be inferred. As we illustrate, by modeling the trace link recovery problem in a probabilistic manner, we are able to construct an an automated approach that largely overcomes the typical drawbacks discussed in \secref{sec:intro} \david{Fix this reference}. To understand this context, let us consider the general definition of Bayes' Theorem:

%\marginnote{
\begin{equation}
P(H|O) = \frac{P(O|H)\cdot P(H)}{P(O)}
\end{equation}
%}

\noindent where $H$ is a hypothesis regarding some phenomenon, $O$ is a set of observations that provide some information about the hypothesis, and where our goal is to infer or estimate the probability that our hypothesis is true $P(H|O)$, which is called the \textit{posterior probability distribution}, or more simply the \textit{posterior}. However, a given hypothesis is rarely made in a vacuum, and one typically holds some \textit{prior belief} as to the probability that is being inferred.  This prior belief is modeled as a probability distribution $P(H)$, which we will simply refer to as the \textit{prior}, and can be influenced by a number of factors. In order for the posterior to be inferred from a set of observations, these must be modeled in a probabilistic manner. This is the purpose of the \textit{likelihood} $P(O|H)$, which is a probability distribution that is derived purely from observed data. Thus, in Bayesian inference initial beliefs are represented as the prior, observations are modeled as the likelihood and the final beliefs are represented by the posterior. This posterior probability distribution can be \textit{inferred} via one of several existing statistical inference techniques. In framing the problem of inferring trace links as a Bayesian problem, we consider our hypothesis to be whether a given trace link exists between a single source artifact $S_x$ and a single target artifact $T_y$. Given the nature of trace links (\eg a link either does or does not exist) we can model our prior as a distribution on the interval $[0,1]$, where 1 indicates the presence of a link and 0 indicates an absence. 

\subsection{A Hierarchical Bayesian Network for Traceability}

In the context of this paper, we will consider our \textit{likelihood} (observations) to be the binary indication that a link exists according to a set of textual similarity measures and an empirically derived threshold value. However, given that we aim to model multiple factors that might influence traceability, our model employs multiple \textit{priors}, called \textit{hyperpriors}, forming a Hierarchical Bayesian Network (HBN). In this work, we consider three priors corresponding to the three factors we wish to model: (i) a normalized set of diverse textual similarity measures, (ii) developer expertise, and (iii) transitive trace links. We assign each of these priors an initial probability distribution, which is then influenced and estimated based upon observable data (e.g. a developer confirming or denying a trace link). Once this network is established, the \textit{posterior} can be computed via one of several estimation techniques. By modeling these three information sources, our technique is able to largely overcome the limitations enumerated in \secref{sec:intro} \david{Fix this reference}. HBNs are also highly extensible via adjustments to the prior(s). Thus our defined model be capable of adapting to advancements in textual similarity measures or considering new development artifacts from future development workflows.


%------------------------------------------------


\section{Deep Code Interpretability Problem}
\label{sec:deep-interpret-problem}

%1. Establishing the importance of the field
The combination of large amounts of freely available code-related data, which can be mined from open source repositories, and ever-more sophisticated Neural Code Model (\nlm) architectures have fueled the development of Software Engineering (SE) tools with increasing effectiveness. \nlms have (seemingly) illustrated promising performance across a range of different SE tasks~\citep{Watson:ICSE20,White:MSR15,ciniselli2021empirical,Mastropaolo2021StudyingTasks}. In particular, \textit{code generation} has been an important area of SE research for decades, enabling tools for downstream tasks such as code completion~\citep{MSR-Completion}, program repair~\citep{Chen2019sequencer}, and test case generation~\citep{Watson:ICSE20}. In addition, industry interest in leveraging \nlms for code generation has also grown as evidenced by tools such as Microsoft's IntelliCode \citep{intellicode}, Tabnine \citep{tabnine}, OpenAI's Codex \citep{openai_codex}, and GitHub's Copilot \citep{github_copilot}. Given the prior popularity of code completion engines within IDEs~\citep{murphy2006ide}, and the pending introduction of, and investment in commercial tools, \nlms for code generation will almost certainly be used to help build production software systems in the near future, if they are not being used already.

%2. Presenting the general problem
However, it is generally accepted that \textit{Neural Language Models} operate in a black-box fashion. That is, we are uncertain how these models \textit{arrive at decisions}, which is why \nlms suffer from \textit{incompleteness} in problem formalization \citep{Doshi-Velez2017TowardsLearning}. As such, much of the work on \nlms has primarily relied upon automated metrics (\eg Accuracy, BLEU, METEOR, ROUGE) as an evaluation standard. Skepticism within the natural language processing (NLP) research community is growing regarding the efficacy of current automated metrics ~\citep{ribeiro2020checklist, rei2020comet, kocmi2021ship}, as they tend to overestimate model performance. Even benchmarks that span multiple tasks and metrics have been shown to lack robustness, leading to incorrect assumptions on model comparisons \citep{dehghani2021benchmark}. 

%3. Previous and/or Current Research
Despite the increasing popularity and apparent effectiveness of neural code generation tools, there is still much that is unknown regarding the practical performance of these models, their ability to learn and predict different code-related concepts, and their current limitations. Some of the most popular models for code generation have been adapted from the field of NLP, and thus may inherit the various limitations often associated with such models --- including biases, memorization, and issues with data inefficiency, to name a few~\citep{bender2021parrots}. In fact, recent work from Chen \etal \citep{chen2021evaluating} illustrates that certain issues, such as alignment failures and biases, do exist for large-scale \nlms. Most of the conclusions from Chen~\etal's study were uncovered through manual analysis, \eg through sourcing counterexamples, making it difficult to rigorously quantify or to systematically apply such an analysis to research prototypes~\citep{wu2019errudite}. Given the rising profile and role that \nlms for code generation play in SE, and the current limitations of adopted evaluation techniques, it is clear that new methods are needed that provide deeper insight into \nlms' performance. Notable work has called for a more systematic approach \citep{ribeiro2020checklist} that aims to understand a given model's behavior according to its linguistic capabilities and tests customized for the given task for which a model is applied.

%4. The GaP (or what is missing). Describe the specific problem. Present a prediction to be tested. 
As the discussion above suggests, while it may appear that \nlms have begun to achieve promising performance, it is clearly insufficient to examine \underline{only} prediction values (\ie the \textit{\textbf{what}} of \nlms' decision). This current status quo, at best, provides an incomplete picture of the limitations and caveats of \nlms for code. %As scientists, we allow theories to be falsifiable by means of empirical observations and proper model explanations to avoid pseudo-scientific claims. 
Given the potential impact and consequence of these models and their resulting applications, there is a clear need to strive for a more complete understanding of how they function in practice. As such, we must push to understand how \nlms arrive at their predictions (\ie the \textit{\textbf{why}} of \nlms' decision). In this paper, we cast this problem of achieving a more complete understanding of \nlms as an \textit{Interpretable Machine Learning} task and posit that we can leverage the theory of \textit{causation} as a mechanism to explain \nlms prediction performance. We hypothesize that this mechanism can serve as a useful debugging tool for detecting biases, understanding limitations, and eventually, increasing the reliability and robustness of \nlms employed for the task of code generation \citep{molnar2019interpret,Doshi-Velez2017TowardsLearning}.

%5.Describing the paper itself
This paper introduces \codegen, a novel post-hoc interpretability method specifically designed for understanding the effectiveness of \nlms. The intention of \codegen is to establish a robust and adaptable methodology for \textit{interpreting} predictions of \nlms trained on code in contrast to simply \textit{measuring} the accuracy of these same \nlms. More specifically, \codegen consists of two major conceptual components, (i) a \textit{structural causal graph}, and (ii) a \textit{causal inference mechanism}. \codegen's \textit{structural causal graph} maps model predictions to programming language (PL) concepts and variations in test datasets at different levels of granularity, thus enabling statistical analyses of model predictions rooted in understandable concepts. While our methodology allows for extensiblility in defining relevant PL concepts, we offer an initial \textit{code taxonomy} as a generalizable example. However, examining statistical properties of model predictions in isolation does not provide \textit{explanations} regarding observed performance. As such, the second theoretical component of \codegen adopts a \textit{causal understanding} mechanism that can explain observed trends in model predictions rooted in the aforementioned programming language concepts. This causal inference mechanism allows for the generation of explanations of model performance rooted in \codegen's understandable concepts. Through the introduction of this interpretability framework, we aim to help SE researchers and practitioners by allowing them to understand the potential limitations of a given model, work towards improving models and datasets based on these limitations, and ultimately make more informed decisions about how to build automated developer tools given a more holistic understanding of \textit{\textbf{what}} \nlms are predicting and \textit{\textbf{why}} the predictions are being made.

%5 Announcing Findings
To showcase the types of insights that \codegen can uncover, we perform a case study on different variations of two popular deep learning architectures for the task of code generation, namely RNNs~\citep{RNNs} and Transformers \citep{vaswani2017transformers} trained on the CodeSearchNet dataset~\citep{husain2019codesearchnet}. We instantiated our study using ten structural code categories derived from the Java programming language. This study resulted in several notable findings illustrating the efficacy of our interpretability technique: (i) we find that our studied models learn to predict tokens related to code blocks (\eg brackets, parentheses, semicolons) more effectively than most other code token types (\eg loops, conditionals, datatypes), (ii) we found that our studied models are sensitive to seemingly subtle changes in code syntax, reinforcing previous studies concluding the same~\citep{rabin2021generalizability}, and (iii) our studied models are only marginally impacted by the presence of comments and bugs, which challenges findings from previous work~\citep{Baishakhi2016buggy}.


%\include{Chapter-Introduction}


% Start calling the chapters Appendices
%----------------------------------------------------------------------------------------
%	APPENDIX PAGES
%----------------------------------------------------------------------------------------
\appendix
\chapter{First Appendix with Awesome Data}
\label{appendixA}

%-----------------------------------
\section{Awesome Appendix Section}
\label{appendixA:sec1}
%-----------------------------------




%\input{appendixB}
%\input{appendixC}


%--List of references not actually cited in the document.
%\nocite{NDSS04DTLS}

%--Include the bibliography
\makeThesisBib{references/bibliographyII}
%\bibliography{references}
%% I did not include the following in my dissertation, so I have no knowledge
%% of their formatting compliance. -- Ruth
%\makeThesisVita{vita}
%%\makeUMIAbstract{abstract}
\end{document}
